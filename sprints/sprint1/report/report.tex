\documentclass[a4paper]{article}

\usepackage[portuguese]{babel}
\usepackage[utf8]{inputenc}
\usepackage[T1]{fontenc}

\newcommand{\documentTitle}{PJAVA Compiler} %Macro definition
\newcommand{\documentAuthors}{João Rafael (2008111876, jprafael@student.dei.uc.pt) \and José Ribeiro (2008112181, jbaia@student.dei.uc.pt)} %Macro definition

\title{\documentTitle}
\author{\documentAuthors{}}

\usepackage{hyperref}
\hypersetup{
	pdftitle = \documentTitle
	,pdfauthor = \documentAuthors
	,pdfsubject = {Project \#1 Report}
	,pdfkeywords = {PJAVA Compiler}
	,pdfborder = {0 0 0}
}

\usepackage{minted}
\usepackage{verbatim} 
\usepackage{subfig}
\usepackage{amsmath}
\usepackage{wrapfig}
\usepackage{array}
\usepackage{anysize}
\usepackage{lscape}
\usepackage[pdftex]{graphicx}
\usepackage{longtable}

\marginsize{3.5cm}{3.5cm}{3cm}{3cm}

\usemintedstyle{ignore}

\makeatletter

\begin{document}
\renewcommand{\figurename}{Figure}
\maketitle
\cleardoublepage

\tableofcontents
\cleardoublepage

\setlength{\parindent}{1cm}
\setlength{\parskip}{0.3cm}

\section{Syntax Analysis}

\indent \indent Tal como indicado no enunciado do projecto, na linguagem PJAVA apenas se pretende explorar a vertente procedimental do JAVA. 
Deste modo, decidimos que implementar \texttt{packages} seria desnecessário.
Devido a esta escolha, o método \texttt{System.out.println()} é implementado como uma função \emph{built-in} (que será distinguida na semântica).

Outras funcionalidades da linguagem que necessitem implicitamente ou explicitamente de um objecto também não foram implementadas (\texttt{try-catch-finally}, \texttt{syncronized}, \\ \texttt{instanceof}, entre outros).

Todas as restantes funcionalidades estão implementadas de acordo com a específicação.

\cleardoublepage

\subsection {Grammar}
\begin{minted}[mathescape]{antlr}
application
	: class_def END
	;

array_decl
	: type_object dims_empty_list
	;

assign_op
	: var '=' expr
	| var SHIFT_R_ASSIGN expr
	| var SHIFT_L_ASSIGN expr
	| var ADD_ASSIGN expr
	| var SUB_ASSIGN expr
	| var MUL_ASSIGN expr
	| var DIV_ASSIGN expr
	| var MOD_ASSIGN expr
	| var AND_ASSIGN expr
	| var XOR_ASSIGN expr
	| var OR_ASSIGN expr
	;

binary_op
	: expr '+' expr
	| expr '-' expr
	| expr '*' expr
	| expr '/' expr
	| expr '%' expr

	| expr '&' expr
	| expr '|' expr
	| expr '^' expr
	| expr SHIFT_R expr
	| expr SHIFT_L expr
	
	| expr AND_OP expr
	| expr OR_OP expr

	| expr EQ_OP expr
	| expr NE_OP expr
	| expr '<' expr
	| expr '>' expr
	| expr LE_OP expr
	| expr GE_OP expr
	| expr EQ3_OP expr
	| expr NE3_OP expr
	| assign_op
	;

break
	: BREAK ';'
	| BREAK ID ';'
	;

class_def
	: CLASS ID '{' '}'
	| CLASS ID '{' class_stmt_list '}'
	;

class_stmt
	: member_stmt
	| class_stmt_privacy member_stmt
	| class_stmt_scope member_stmt
	| class_stmt_privacy class_stmt_scope member_stmt
	;

class_stmt_privacy
	: PUBLIC
	| PRIVATE
	;

class_stmt_scope
	: FINAL
	| STATIC
	| STATIC FINAL
	| FINAL STATIC
	;

class_stmt_list
	: class_stmt
	| class_stmt class_stmt_list
	;

continue
	: CONTINUE ';'
	| CONTINUE ID ';'
	;

dims
	: dims_sized_list
	| dims_sized_list dims_empty_list
	;

dims_empty
	: '[' ']'
	;

\end{minted}

\cleardoublepage

\begin{minted}[mathescape]{antlr}
dims_empty_list
	: dims_empty
	| dims_empty_list dims_empty
	;

dims_sized
	: '[' expr ']'
	;

dims_sized_list 
	: dims_sized
	| dims_sized_list dims_sized
	; 

do_while
	: DO stmt WHILE '(' expr ')' ';'
	;

expr
	: var %prec LOW_PREC
	| new_op %prec LOW_PREC
	| '(' expr ')'
	| '(' type_decl ')' expr
	| CONSTANT
	| func_call
	| expr_op
	;

expr_list
	: expr
	| expr ',' expr_list
	;

expr_op
	: unary_op
	| binary_op
	| ternary_op
	;
for
	: FOR '(' for_init ';' for_cond ';' for_inc ')' stmt
	;

for_cond
	: /* empty */
	| expr
	;

\end{minted}

\cleardoublepage

\begin{minted}[mathescape]{antlr}
for_expr
	: incr_op
	| assign_op
	| func_call
	;

for_expr_list
	: for_expr
	| for_expr ',' for_expr_list
	;

for_init
	: /* empty */
	| var_defs
	| for_expr_list
	;

for_inc
	: /* empty */
	| for_expr_list
	;
 	
func_call
	: ID func_call_arg_list
	;

func_call_arg_list
	: '(' ')'
	| '(' expr_list ')'
	;

func_def
	: type_decl ID func_def_args '{' '}'
	| type_decl ID func_def_args '{' stmt_list '}'
	;

func_def_arg
	: type_decl ID
	;

func_def_arg_list
	: func_def_arg
	| func_def_arg ',' func_def_arg_list
	;

func_def_args
	: '(' ')'
	| '(' func_def_arg_list ')'
	;

if
	: IF '(' expr ')' stmt %prec LOW_PREC
	| IF '(' expr ')' stmt ELSE stmt
	;

incr_op
	: INC_OP var
	| var INC_OP
	| DEC_OP var
	| var DEC_OP
	;

loop_stmt
	: for
	| ID ':' for
	| while
	| ID ':' while
	| do_while
	| ID ':' do_while
	;

member_stmt
	: ';'
	| var_stmt
	| func_def
	;

new_op
	: NEW type_object dims
	;

return
	: RETURN ';'
	| RETURN expr ';'
	;

\end{minted}

\cleardoublepage

\begin{minted}[mathescape]{antlr}
stmt
	: ';'
	| '{' '}'
	| '{' stmt_list '}'
	| var_stmt
	| assign_op ';'
	| incr_op ';'
	| if
	| loop_stmt
	| func_call ';'
	| switch
	| break
	| continue
	| return
	;

stmt_list
	: stmt
	| stmt stmt_list
	;

switch
	: SWITCH '(' expr ')' '{' '}'
	| SWITCH '(' expr ')' '{' switch_stmt_list '}'
	;

switch_stmt
	: DEFAULT ':'
	| DEFAULT ':' stmt_list
	| CASE CONSTANT ':'
	| CASE CONSTANT ':' stmt_list
	;

switch_stmt_list
	: switch_stmt
	| switch_stmt switch_stmt_list
	;

ternary_op
	: expr '?' expr ':' expr
	;

type_decl
	: type_object
	| array_decl
	;

\end{minted}

\cleardoublepage

\begin{minted}[mathescape]{antlr}
type_native
	: BOOL
	| BYTE
	| CHAR
	| DOUBLE
	| FLOAT
	| INT
	| LONG
	| SHORT
	| STRING
	| VOID
	;

type_object
	: type_native
	;

unary_op
	: incr_op
	| '+' expr
	| '-' expr
	| '!' expr
	| '~' expr
	;

var
	: ID %prec LOW_PREC
	| '(' var ')'
	| '(' new_op ')'
	| var dims_sized
	| func_call dims_sized
	;

var_def
	: var_def_left
	| var_def_left '=' var_initializer
	;

var_def_list
	: var_def
	| var_def ',' var_def_list
	;

var_def_left
	: ID
	| ID dims
	| ID dims_empty_list
	;

\end{minted}

\cleardoublepage

\begin{minted}[mathescape]{antlr}
var_defs
	: type_decl var_def_list
	;

var_initializer
	: '{' '}'
	| '{' var_initializer_list '}'
	| '{' var_initializer_list ',' '}'
	| expr
	;

var_initializer_list
	: var_initializer
	| var_initializer_list ',' var_initializer
	;

var_stmt															
	: var_defs ';'
	;

while
	: WHILE '(' expr ')' stmt
	;
\end{minted}

\cleardoublepage

\subsection{Abstract Syntax}

\indent \indent A partir da definição da gramática é possível extrair a sintaxe abstracta, subdividindo cada disjunção de regras numa disjunção de predicados.
Aos nomes das regras acrescentamos o prefixo "is\_" ao seu predicado correspondente;
dessa mesma forma, acrescentamos o sufixo "\_list" às regras que constituem listas de elementos.
A disjunção dessas regras (no YACC representada pelo símbolo |) é transformada na disjunção de predicados utilizando para isso
o símbolo de disjunção lógico $\vee$.

\indent Por exemplo, na regra
\begin{minted}[mathescape]{antlr}
type_decl
	: type_object
	| array_decl
	;
\end{minted}

o seu correspondente na sintaxe abstracta é \\ \\
is\_type\_decl $\to$ <is\_type\_object : type\_object> $\vee$ <is\_array\_decl : array> \\

\indent \indent Apesar de aparecerem referidos na definição da sintaxe abstracta, is\_ID, \\
is\_CONSTANT e outros tipos (identificados com a mesma nomenclatura),
não são expressos segundo uma regra pois são tokens obtidos a partir da análise lexical do programa fonte.

Segue-se a especificação da sintaxe abstracta.

\cleardoublepage

\subsubsection{Abstract Syntax Specification}
\begin{longtable}{lcl}
	is\_application				& $\to$ & <is\_class\_def : class> \\
	is\_array\_decl 			& $\to$ & <is\_type\_object : type> <is\_dims\_empty\_list : empty> \\
	is\_assign\_op				& $\to$ & <is\_var : var> <is\_type\_assign\_op : op> <is\_expr : expr> \\
	is\_type\_assign\_op		& $\to$ & <is\_EQUAL> \\
								& $\vee$ & <is\_SHIFT\_R\_ASSIGN> \\
								& $\vee$ & <is\_SHIFT\_L\_ASSIGN> \\
								& $\vee$ & <is\_ADD\_ASSIGN> \\
								& $\vee$ & <is\_SUB\_ASSIGM> \\
								& $\vee$ & <is\_MUL\_ASSIGN> \\
								& $\vee$ & <is\_DIV\_ASSIGN> \\
								& $\vee$ & <is\_MOD\_ASSIGN> \\
								& $\vee$ & <is\_AND\_ASSIGN> \\
								& $\vee$ & <is\_XOR\_ASSIGN> \\
								& $\vee$ & <is\_OR\_ASSIGN> \\
	is\_binary\_op				& $\to$ & <is\_assign\_op : assign> \\
								& $\vee$ & (<is\_expr : left> <is\_type\_binary\_op : oper> <is\_expr : right>) \\
	is\_type\_binary\_op		& $\to$ & <is\_ADD> \\
								& $\vee$ & <is\_SUB> \\
								& $\vee$ & <is\_MUL> \\
								& $\vee$ & <is\_DIV> \\
								& $\vee$ & <is\_MOD> \\
								& $\vee$ & <is\_AND> \\
								& $\vee$ & <is\_OR> \\
								& $\vee$ & <is\_XOR> \\
								& $\vee$ & <is\_SHIFT\_R>\\
								& $\vee$ & <is\_SHIFT\_L> \\
								& $\vee$ & <is\_LOGIC\_AND>\\
								& $\vee$ & <is\_LOGIC\_OR> \\
								& $\vee$ & <is\_LESS> \\
								& $\vee$ & <is\_GREAT> \\
								& $\vee$ & <is\_LESS\_EQ> \\
								& $\vee$ & <is\_GREAT\_EQ> \\
								& $\vee$ & <is\_EQ3> \\
								& $\vee$ & <is\_NEQ3> \\
	is\_break 					& $\to$ & $\lambda$ $\vee$ <is\_ID : label> \\
	is\_class\_def				& $\to$ & <is\_ID : id> <is\_class\_stmt\_list : body> \\
	is\_class\_stmt\_privacy	& $\to$ & is\_PRIVATE $\vee$ <is\_PUBLIC>\\	
	is\_class\_stmt\_scope		& $\to$ &($\lambda$ $\vee$ <is\_FINAL>) ($\lambda$ $\vee$ <is\_STATIC>) \\
	is\_class\_stmt 			& $\to$ & <is\_class\_stmt\_scope : scope> <is\_member\_stmt : stmt> \\
	is\_continue				& $\to$ & $\lambda$ $\vee$ <is\_ID : label> \\
	is\_dims					& $\to$ & <is\_dims\_sized\_list : sized> ($\lambda$ $\vee$ <is\_dims\_empty\_list : empty>) \\
	is\_dims\_sized				& $\to$ & <is\_expr : expr> \\
	is\_do\_while				& $\to$ & <is\_stmt : body> <is\_expr : cond> \\
	is\_expr					& $\to$ & <is\_var : var> \\
								& $\vee$ & <is\_new\_op : new\_op> \\
								& $\vee$ & (<is\_type\_decl : type> <is\_expr : expr>) \\
								& $\vee$ & <is\_CONSTANT : constant> \\
								& $\vee$ & <is\_func\_call : func\_call> \\
								& $\vee$ & <is\_expr\_op : operation> \\

	is\_expr\_op				& $\to$ & <is\_unary\_op : unary> \\
								& $\vee$ & <is\_binary\_op : binary>\\
								& $\vee$ & <is\_ternary\_op : ternary>\\
	is\_for						& $\to$ & ($\lambda$ $\vee$ <is\_for\_init : init>) ($\lambda$ $\vee$ <is\_for\_cond : cond>) $\cdots$ \\
								& $\cdots$ & ($\lambda$ $\vee$ <is\_for\_inc : inc>) <is\_stmt : body> \\
	is\_for\_cond				& $\to$ & $\lambda$ $\vee$ <is\_expr : expr> \\	
	is\_for\_expr				& $\to$ & <is\_incr\_op : incr> \\
								& $\vee$ & <is\_assign\_op : assign> \\
								& $\vee$ & <is\_func\_call : func\_call> \\
	is\_for\_inc				& $\to$ & <is\_for\_expr\_list : list> \\ 	
	is\_for\_init				& $\to$ & <is\_var\_defs : var\_defs> $\vee$ <is\_for\_expr\_list: expr\_list> \\
	is\_func\_call				& $\to$ & <is\_ID : id> ($\lambda$ $\vee$ <is\_func\_call\_arg\_list : args>) \\
	is\_func\_call\_arg\_list	& $\to$ & <is\_expr\_list : expr\_list> \\	
	is\_func\_def				& $\to$ & <is\_type\_decl : type> <is\_ID : id> \\
								&       & <is\_func\_def\_args : args> <is\_stmt\_list : body> \\
	is\_func\_def\_arg			& $\to$ & <is\_type\_decl : type> <is\_ID : id> \\
	is\_func\_def\_args			& $\to$ & <is\_func\_def\_arg\_list : list> \\		
	is\_if						& $\to$ & <is\_expr : cond> <is\_stmt : then\_body> ($\lambda$ $\vee$ <is\_stmt : else\_body>) \\
	is\_incr\_op				& $\to$ &(<is\_incr\_op\_operator : operator> <is\_var : var>) \\
								& $\vee$ & (<is\_var : var> <is\_incr\_op\_operator : operator>) \\
	is\_incr\_op\_operator		& $\to$ & <is\_INC> $\vee$ <is\_DEC> \\
	is\_loop\_stmt				& $\to$ & ($\lambda$ $\vee$ <is\_ID : loop\_label>)\\
								& ( & <is\_for : for\_stmt> \\
								& $\vee$ & <is\_while : while\_stmt> \\ 
								& $\vee$ & <is\_do\_while : do\_while\_stmt> \\
								& ) & \\
	is\_member\_stmt			& $\to$ & <is\_var\_stmt : var> $\vee$ <is\_func\_def : func\_def> \\
	is\_new\_op					& $\to$ & <is\_type\_object : type\_object> <is\_dims : dims> \\
	is\_return					& $\to$ & $\lambda$ $\vee$ <is\_expr : value> \\
	is\_stmt					& $\to$ & <is\_stmt\_list : stmt\_list> \\
								& $\vee$ & <is\_var\_stmt : var> \\
								& $\vee$ & <is\_assign\_op : assign> \\
								& $\vee$ & <is\_incr\_op : incr> \\
								& $\vee$ & <is\_if : if\_stmt> \\
								& $\vee$ & <is\_loop\_stmt : loop> \\
								& $\vee$ & <is\_func\_call : func\_call> \\
								& $\vee$ & <is\_switch : switch\_stmt> \\
								& $\vee$ & <is\_break : break\_stmt> \\
								& $\vee$ & <is\_continue : continue\_stmt> \\
								& $\vee$ & <is\_return : return\_stmt> \\
	is\_switch					& $\to$ & <is\_expr : expr> <is\_switch\_stmt\_list : list> \\
	is\_switch\_stmt			& $\to$ & <is\_stmt\_list : list> ($\lambda$ $\vee$ <is\_CONSTANT : constant> \\
	is\_ternary\_op				& $\to$ & <is\_expr : if\_expr> <is\_expr : then\_expr> <is\_expr : else\_expr> \\
	is\_type\_decl				& $\to$ & <is\_type\_object : type\_object> $\vee$ <is\_array\_decl : array> \\
	is\_type\_native			& $\to$ & <is\_BOOL> \\
								& $\vee$ & <is\_BYTE> \\
								& $\vee$ & <is\_CHAR> \\
								& $\vee$ & <is\_DOUBLE> \\
								& $\vee$ & <is\_FLOAT> \\
								& $\vee$ & <is\_INT> \\
								& $\vee$ & <is\_LONG> \\
								& $\vee$ & <is\_SHORT> \\
								& $\vee$ & <is\_STRING> \\
								& $\vee$ & <is\_VOID> \\
	is\_type\_object			& $\to$ & <is\_type\_native: type> \\
	is\_unary\_op				& $\to$ & <is\_incr\_op: incr> $\vee$ (<is\_unary\_op\_operator: op> <is\_expr: expr>) \\
	is\_unary\_op\_operator		& $\to$ & <is\_PLUS> \\
								& $\vee$ & <is\_MINUS> \\
								& $\vee$ & <is\_NOT> \\
								& $\vee$ & <is\_BIN\_NOT> \\
	is\_var					 	& $\to$ & <is\_ID: id> \\
								& $\vee$ & <is\_new\_op: new\_op> \\
								& $\vee$ & (<is\_var : var> <is\_dims\_sized : dims>) \\
								& $\vee$ & (<is\_func\_call : func\_call> <is\_dims\_sized : dims>) \\
	is\_var\_def			 	& $\to$ & <is\_var\_def\_left: left> ($\lambda$ $\vee$ <is\_var\_initializer : var\_init>) \\
	is\_var\_def\_left		 	& $\to$ & <is\_ID: id> (<is\_dims : dims> $\vee$ is\_dims\_empty\_list : empty) \\
	is\_var\_defs		 		& $\to$ & <is\_type\_decl: type> <is\_var\_def\_list : list> \\
	is\_var\_initializer 		& $\to$ & <is\_var\_initializer\_list: array> $\vee$ <is\_expr : expr> \\
	is\_var\_stmt				& $\to$ & <is\_var\_defs: var\_defs> \\	
	is\_while 					& $\to$ & <is\_expr : cond> <is\_stmt : body> \\
\end{longtable}

\cleardoublepage

\subsection{Abstract Syntax Tree Implementation}
A implementação das estruturas da árvore de sintaxe abstracta foi feita segundo as seguintes regras:
\begin{itemize}
	\item Cada conjunção de predicados é colocada numa \texttt{struct};
	\item Cada disjunção, de ou predicados ou conjunções de predicados, é colocada numa \texttt{union},
		utilizando um \texttt{enum} auxiliar de identificação do campo da \texttt{union} utilizado
		(nalguns casos determinando implicitamente o valor de um campo);
	\item Todas os predicados que seleccionam apenas objectos elementares são \texttt{typedef}'d para o seu objecto elementar correspondente,
		dado que estas regras foram mantidas com o objectivo de facilitar futuras extensões sintácticas da linguagem.
\end{itemize}

A regra is\_empty\_dims é omitida da AS uma vez que a informação semântica que esta indica não depende do seu conteúdo, mas sim da sua existência.
Assim, na implementação de is\_empty\_dims\_list é apenas necessário guardar um inteiro, correspondente ao número de vezes que a regra é repetida.

\end{document}
